\documentclass{article}

\usepackage[utf8]{inputenc}
\usepackage{libertine}
\renewcommand{\familydefault}{\sfdefault}
\usepackage[T1]{fontenc}
\usepackage[cjk]{kotex} % [cjk] option is needed for texlive 2022 engine (see more: https://www.overleaf.com/blog/tex-live-2022-now-available)

\usepackage{mathrsfs}
\usepackage{amsfonts, amsmath, amssymb, amsthm}
\usepackage{calligra}
\usepackage{mathtools}
\usepackage{mathabx} % Needed to use \widecheck
\usepackage{quiver}

% \usepackage{fullpage}
\usepackage{enumitem}
\setlist[enumerate]{label=(\alph*)}
\usepackage[most]{tcolorbox}

\usepackage{makecell}
\usepackage{hhline}

\usepackage{graphicx}
\usepackage{caption}
\usepackage{subcaption}
\usepackage{svg}

\usepackage{hyperref}
\hypersetup{
    colorlinks=true,
    linkcolor=blue,
    filecolor=magenta,
    urlcolor=orange,
}
\usepackage{ulem}

%%%%%%%%%%%%%%%%%%%%%%%%%%%%%%%%%%%%%%%%
\setlength{\parskip}{\baselineskip}

\newcommand{\separater}{\begin{center} \rule{0.5\linewidth}{0.3pt} \end{center}}

\usepackage{tikz}
\newcommand*\circled[1]{\tikz[baseline=(char.base)]{\node[shape=circle,draw,inner sep=2pt] (char) {#1};}} % Circled numbering, https://tex.stackexchange.com/a/7045

% \renewcommand{\thesubsection}{\arabic{subsection}} % Subsection numbering

%%%%%%% Theorem Definition Style %%%%%%%
\theoremstyle{plain}
\newtheorem{lem}{Lemma}
\newtheorem{fact}[lem]{Fact}
\newtheorem{prop}[lem]{Proposition}
\newtheorem{thm}[lem]{Theorem}
\newtheorem{thmdef}[lem]{Theorem-Definition}
\newtheorem{cor}[lem]{Corollary}

\theoremstyle{definition}
\newtheorem{defn}[lem]{Definition}
\newtheorem{exam}[lem]{Examples}
\newtheorem*{exer}{Exercise}
\newtheorem*{prob}{Problem}

%%%%%%%%%%%%%%%%%%%%%%%%%%%%%%%%%%%%%%%%
\renewcommand{\setminus}{\smallsetminus}
\renewcommand{\sl}{\mathfrak{sl}}

\DeclareMathOperator{\Ab}{\underline{\textsf{Ab}}}
\DeclareMathOperator{\ad}{ad}
\DeclareMathOperator{\Ad}{Ad}
\DeclareMathOperator{\bdry}{\partial}
\DeclareMathOperator{\coker}{coker}
\DeclareMathOperator{\CP}{\mathbb{C}\mathrm{P}}
\DeclareMathOperator{\ch}{char}
\DeclareMathOperator{\eval}{eval}
\DeclareMathOperator{\Ext}{Ext}
\DeclareMathOperator{\GL}{GL}
\DeclareMathOperator{\Gr}{Gr}
\DeclareMathOperator{\Hom}{Hom}
\DeclareMathOperator{\calHom}{\mathscr{H}\text{\kern -3pt {\calligra\large om}}\,}
\DeclareMathOperator{\id}{id}
\DeclareMathOperator{\im}{im}
\DeclareMathOperator{\PGr}{\mathbf{Gr}}
\DeclareMathOperator{\Pic}{Pic}
\DeclareMathOperator{\PSh}{\underline{\textsf{PSh}}}
\DeclareMathOperator{\red}{red}
\DeclareMathOperator{\rk}{rank}
\DeclareMathOperator{\Sch}{\underline{\textsf{Sch}}}
\DeclareMathOperator{\RP}{\mathbb{R}\mathrm{P}}
\DeclareMathOperator{\Sh}{\underline{\textsf{Sh}}}
\DeclareMathOperator{\SL}{SL}
\DeclareMathOperator{\Sym}{Sym}
\DeclareMathOperator{\Span}{Span}
\DeclareMathOperator{\Spec}{Spec}
\DeclareMathOperator{\Supp}{Supp}
\DeclareMathOperator{\Tor}{Tor}
\DeclareMathOperator{\Tors}{Tors}

%%%%%%%%%%%%%%%%%%%%%%%%%%%%%%%%%%%%%%%%

\title{\textbf{Harthshorne Reading Study}\\[0.5em] {\large at KAIST Mathematical Problem Solving Group}}
\author{20190262 Jeongwoo Park}
\date{Period: 2022 December - 2023 January}

\begin{document}

\maketitle

\separater

\section*{Section II.1}

\begin{exer}[II.1.8]
    For any open subset $U \subset X$, show that the functor $\Gamma(U, \underline{\hspace{0.6em}})$ from sheaves on $X$ to abelian groups is a left-exact functor. The functor $\Gamma(U, \underline{\hspace{0.6em}})$ need not be exact.
\end{exer}

\begin{tcolorbox}[breakable, enhanced]
    \textbf{First Solution} (for beginners). The sequence
    % https://q.uiver.app/?q=WzAsNCxbMCwwLCIwIl0sWzEsMCwiXFxtYXRoc2Nye0Z9JyJdLFsyLDAsIlxcbWF0aHNjcntGfSJdLFszLDAsIlxcbWF0aHNjcntGfScnIl0sWzEsMl0sWzIsMywiZyJdLFswLDFdXQ==
    \[\begin{tikzcd}
        0 & {\mathscr{F}'} & {\mathscr{F}} & {\mathscr{F}''}
        \arrow[from=1-2, to=1-3]
        \arrow["g", from=1-3, to=1-4]
        \arrow[from=1-1, to=1-2]
    \end{tikzcd}\]
    is exact means $\mathscr{F}$ is the kernel object of the second map $g$, i.e, there is an isomorphism $\alpha: \mathscr{F}' \to \ker g$ such that
    % https://q.uiver.app/?q=WzAsNCxbMCwwLCJcXG1hdGhzY3J7Rn0nIl0sWzEsMCwiXFxtYXRoc2Nye0Z9Il0sWzIsMCwiXFxtYXRoc2Nye0Z9JyciXSxbMSwxLCJcXGtlciBnIl0sWzAsMV0sWzEsMiwiZyJdLFszLDEsIiIsMCx7InN0eWxlIjp7InRhaWwiOnsibmFtZSI6Imhvb2siLCJzaWRlIjoiYm90dG9tIn19fV0sWzAsMywiXFxzaW0iLDJdXQ==
    \[\begin{tikzcd}
        {\mathscr{F}'} & {\mathscr{F}} & {\mathscr{F}''} \\
        & {\ker g}
        \arrow[from=1-1, to=1-2]
        \arrow["g", from=1-2, to=1-3]
        \arrow[hook', from=2-2, to=1-2]
        \arrow["\sim"', from=1-1, to=2-2]
    \end{tikzcd}\]
    commutes. Taking the global section gives a commutative diagram
    % https://q.uiver.app/?q=WzAsNCxbMCwwLCJcXG1hdGhzY3J7Rn0nKFgpIl0sWzEsMCwiXFxtYXRoc2Nye0Z9KFgpIl0sWzIsMCwiXFxtYXRoc2Nye0Z9JycoWCkiXSxbMSwxLCIoXFxrZXIgZykoWCkiXSxbMCwxXSxbMSwyLCJnX1giXSxbMywxLCIiLDAseyJzdHlsZSI6eyJ0YWlsIjp7Im5hbWUiOiJob29rIiwic2lkZSI6ImJvdHRvbSJ9fX1dLFswLDMsIlxcc2ltIiwyXV0=
    \[\begin{tikzcd}
        {\mathscr{F}'(X)} & {\mathscr{F}(X)} & {\mathscr{F}''(X)} \\
        & {(\ker g)(X)}
        \arrow[from=1-1, to=1-2]
        \arrow["{g_X}", from=1-2, to=1-3]
        \arrow[hook', from=2-2, to=1-2]
        \arrow["\sim"', from=1-1, to=2-2]
    \end{tikzcd}\]
    Since the global section functor and the kernel commutes, we know that $(\ker g)(X) = \ker g_X$. Hence, $\mathscr{F}'(X)$ is a kernel object of the map $g_X$, so the sequence
    % https://q.uiver.app/?q=WzAsNCxbMSwwLCJcXG1hdGhzY3J7Rn0nKFgpIl0sWzIsMCwiXFxtYXRoc2Nye0Z9KFgpIl0sWzMsMCwiXFxtYXRoc2Nye0Z9JycoWCkiXSxbMCwwLCIwIl0sWzAsMV0sWzEsMiwiZ19YIl0sWzMsMF1d
    \[\begin{tikzcd}
        0 & {\mathscr{F}'(X)} & {\mathscr{F}(X)} & {\mathscr{F}''(X)}
        \arrow[from=1-2, to=1-3]
        \arrow["{g_X}", from=1-3, to=1-4]
        \arrow[from=1-1, to=1-2]
    \end{tikzcd}\]
    is exact. \qed
\end{tcolorbox}

\begin{tcolorbox}[breakable, enhanced]
    \textbf{Second Solution} (keep in mind). A functor is exact if and only it it commutes with the kernel. \qed
\end{tcolorbox}

\begin{tcolorbox}[breakable, enhanced]
    \textbf{Third Solution} (for those who are interested in category theory).

    \begin{enumerate}[label=\fbox{\arabic*}]
        \item The global section functor $\PSh \to \Ab$ is exact.
        \item The forgetful functor $\Sh \to \PSh$ is left-exact. Indeed, it is right adjoint to the sheafification functor $\PSh \to \Sh$.
    \end{enumerate}

    By combining these two results, the global section functor $\Sh \to \Ab$ is left-exact. \qed
\end{tcolorbox}

\separater

[[Exercise II.1.16 solution]]

\newpage

\section*{Problems}

\subsection*{Section II.1}

\begin{enumerate}[label = \arabic*.]
    \item Let
    % https://q.uiver.app/?q=WzAsNSxbMCwwLCIwIl0sWzEsMCwiXFxtYXRoc2Nye0Z9JyJdLFsyLDAsIlxcbWF0aHNjcntGfSJdLFszLDAsIlxcbWF0aHNjcntGfScnIl0sWzQsMCwiMCJdLFswLDFdLFsxLDJdLFsyLDNdLFszLDRdXQ==
    \[\begin{tikzcd}
        0 & {\mathscr{F}'} & {\mathscr{F}} & {\mathscr{F}''} & 0
        \arrow[from=1-1, to=1-2]
        \arrow[from=1-2, to=1-3]
        \arrow[from=1-3, to=1-4]
        \arrow[from=1-4, to=1-5]
    \end{tikzcd}\]
    be a sequence of sheaves.
    \begin{enumerate}
        \item Explain how's different the exactness of the sequence in $\Sh$ and in $\PSh$. Do the exactness in one category implies one in the another category?
        \item (With some knowledge in cohomology theory) Why the exactness of the global section is guaranteed only by the first term $\mathscr{F}'$?
        \item Why the sequence
        % https://q.uiver.app/?q=WzAsNSxbMCwwLCIwIl0sWzEsMCwiXFx1bmRlcmxpbmV7XFxtYXRoYmJ7Wn19Il0sWzIsMCwiXFxtYXRoc2Nye0h9Il0sWzMsMCwiXFxtYXRoc2Nye0h9XlxcYXN0Il0sWzQsMCwiMCJdLFswLDFdLFsxLDJdLFsyLDNdLFszLDRdXQ==
        \[\begin{tikzcd}
            0 & {\underline{\mathbb{Z}}} & {\mathscr{H}} & {\mathscr{H}^\ast} & 0
            \arrow[from=1-1, to=1-2]
            \arrow[from=1-2, to=1-3]
            \arrow[from=1-3, to=1-4]
            \arrow[from=1-4, to=1-5]
        \end{tikzcd}\]
        is \emph{not exact} after taking the global section functor? How this fact is related with the sheaf cohomology?
        
        (Here, $\mathscr{H} \in \Sh(\mathbb{C})$ is the sheaf of holomorphic functions, and $\mathscr{H}^\ast \in \Sh(\mathbb{C})$ is one of \emph{invertiable} holomorphic functions.)
    \end{enumerate}
\end{enumerate}

\newpage

\section*{Properties in Philosophy}

\begin{tcolorbox}[breakable, enhanced]
    \begin{prop}(adjoint functors)
        Following pairs are adjoint.
        \begin{enumerate}[label = \arabic*.]
            \item $\text{Sheafification} \dashv \text{Forgetful functor}: \PSh \rightleftarrows \Sh$
            \item $\text{Constant presheaf functor} \dashv \text{Global section functor}: \Ab \rightleftarrows \PSh$
            \item[1+2.] $\text{Constant sheaf functor} \dashv \text{Global section functor}: \Ab \rightleftarrows \Sh$
            \item $\text{Forgetful functor} \dashv \text{Reduction functor}: \Sch \rightleftarrows \Sch_{\red}$
        \end{enumerate}
    \end{prop}
\end{tcolorbox}

[[adjunction and pullback/pushout? (co)limit? commutes... Gluing? Refuction of schemes?]]

\subsection*{Section II.1}

\begin{enumerate}[label = \arabic*.]
    \item An abelian functor is left-exact (resp. right-exact) if and only if it commutes with the kernel (resp. cokernel).
\end{enumerate}

\end{document}